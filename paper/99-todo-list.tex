

% \newpage
% \clearpage


% \section{TODO list}

% explanation: \todo{I am a todo item!}, \tododone{I am already done!}

% \subsection{Comments of the committee}

% 1. Better \textbf{position the paper:} remove any/all claims that this method is the first to consider an approximation objective, including (but not limited to) lines 30-33 and 174-175, and give more credit to previous works on developable surface approximation that have considered approximation error.

% \noindent
% \tododone{- rework related work as this gets our claims straight}\\
% \tododone{- rework positioning of the paper, introduction}\\
% \tododone{- Fig 1: add geodesics and errors}\\
% .

% \noindent
% 2. Add \textbf{seam smoothing} step: integrate post-processing for boundary straightening similar to Julius et al.2005 or even align the seams to salient mesh features (e.g., ridges and valleys) to yield more aesthetically pleasing patches.

% \noindent
% \tododone{- Philipp is working on seams}\\
% \tododone{- re-produce results}\\
% \tododone{- re-fabricate results}\\
% .


% \noindent
% 3. Add more results and \textbf{evaluations:}\\
% 3.1. \textbf{Comparison to previous methods} [Metani and Suzuki] and [Stein et al], especially on more complex models, in a quantitative manner

% \noindent
% \tododone{- work out comparison to related work}\\
% \tododone{- ask Mitani for bunny --> asked, received}\\
% \tododone{- ask Shatz for bunny,  --> asked, cant find them}\\
% \tododone{- ask Massarwi for venus \& hummingbird,  --> asked, is checking}\\
% \tododone{- run Stein bunny}\\
% .

% \noindent
% 3.2. Results to show that the method can handle \textbf{creases and sharp edges} well: not just the cube, but also more realistic models like the fandisk and other mechanical parts

% \noindent
% \tododone{- show approximation steps of cube}\\
% \tododone{- show fandisk}\\
% \tododone{- explain that other methods that used segmentation based on creases are better suited (faster) for such object than our general purpose method.}\\
% .

% \noindent
% 3.3. Experiments to show how sensitive the results are with respect to the \textbf{variation of parameters}, discuss the possibility of offering an intuitive or direct control over the quality of the results (e.g., number, size, or layout of patches)

% \noindent
% \todo{- plan results for this}\\
% .

% \noindent
% 3.4. Experiments to justify the design choices, for example, why DOGs are a good choice relative to other alternatives. 

% \noindent
% \tododone{- show pavilion in fig 2 wrapped with a single DOG}\\
% .

% \noindent
% 3.5. Experiments to show the \textbf{robustness to mesh quality} - show examples where a low-quality input mesh is used for the complete pipeline (not just for the partitioning, as in Fig. 2). Include failure cases where the final projection works badly due to meshes that are too coarse or have too few/badly shaped elements.

% \noindent
% \todo{- additionally show dog model as downloaded from thingiverse}\\
% \tododone{- show failure case: cylinder with vertices only at creases}




% \subsection{other TODOs from reviews/rebuttal}

% \todo{- show overview of pipeline}\\
% \todo{- report used parameters for each result}\\
% \todo{- add image to clarify how $\delta_{distance}$ controls stretch}\\
% \todo{- add image to clarify geodesics sampling and stopping}\\
% % \todo{- add images for clarity (nonlinear projection, geodesics tracing, ...)}\\
% \todo{- clarify that coordinate curve on DOG is a discrete geodesic (R4)}\\
% \tododone{- cite principal curvature estimation: via quadric fitting [Panozzo, Puppo, Rocca. 2010]}\\
% \tododone{- add references from R5}\\
% \tododone{- clarify 'penalty based method' and 'Eproj' (R3) --> sync. with Philipp}\\
% \todo{- add appendix with DOG energies we use}\\




% \newpage
% \clearpage
